\documentclass{scrreprt}
\usepackage{listings}
\usepackage{underscore}
\usepackage{graphicx}
\usepackage[bookmarks=true]{hyperref}
\usepackage[utf8]{inputenc}
\usepackage[english]{babel}
\hypersetup{
    bookmarks=false,    % show bookmarks bar?
    pdftitle={Software Requirement Specification},    % title
    pdfauthor={Galucci, Hernandez, Trepin},                     % author
    pdfsubject={TeX and LaTeX},                        % subject of the document
    pdfkeywords={TeX, LaTeX, graphics, images}, % list of keywords
    colorlinks=true,       % false: boxed links; true: colored links
    linkcolor=blue,       % color of internal links
    citecolor=black,       % color of links to bibliography
    filecolor=black,        % color of file links
    urlcolor=purple,        % color of external links
    linktoc=page            % only page is linked
}%
\def\myversion{1.0 }
\date{}
%\title
\usepackage{hyperref}
\begin{document}

\begin{flushright}
    \rule{16cm}{5pt}\vskip1cm
    \begin{bfseries}
        \Huge{SOFTWARE REQUIREMENTS\\ SPECIFICATION}\\
        \vspace{1.5cm}
        for\\
        \vspace{1.5cm}
        Geogef\\
        \vspace{1.5cm}
        \LARGE{Version \myversion}\\
        \vspace{1.5cm}
        Esteban Galucci \\
        Florencia Hernandez \\
        Gonzalo Trepin \\
        \today\\
    \end{bfseries}
\end{flushright}

\tableofcontents

\chapter{Introduction}

\section{Purpose}

Our main aim is to make Geography exciting and fun to learn about the world we live in. We want to motivate people to study regularly and learn through interactive lessons and quizzes. Our goal is to help users understand different countries, capitals, landforms, flags, and cultures.

\section{Intended Audience and Reading Suggestions}
This SRS is for developers, project managers, users and testers. 

\section{Project Scope}
Geogef offers a dynamic learning platform with different learning trees, interactive lessons, points-based rewards, and comprehensive user statistics.

\chapter{Overall Description}

\section{Product Perspective}
Geogef provides a user-friendly interface for exploring various Geography topics and tracking learning progress. It replaces traditional learning methods with an engaging digital platform.

\section{User Classes and Characteristics}
Geogef caters to users interested in Geography education, offering a gamified learning experience with interactive lessons and progress tracking.

\section{Product Functions}
Geogef functions similarly to popular language learning apps, offering learning trees with different topics, interactive lessons, points and rewards, and detailed user statistics.

\section{Operating Environment}
Geogef operates across multiple platforms, including web browsers on desktop and mobile devices, ensuring accessibility and flexibility for users.

\section{Design}
Geogef's design emphasizes user engagement and ease of use, with intuitive navigation, visually appealing interfaces, and seamless integration of gamified elements.

\chapter{System Features}

\section{Description and Priority}
Key features of Geogef include Learning Trees, Lessons, Points and Rewards, and User Statistics, all aimed at providing an immersive and rewarding learning experience.

\section{Functional Requirements}
\begin{enumerate}
    \item Allow users to register and access the application through a login system.
    \item Inquire about the topics to be learned in the different units.
    \item Enable users to advance to the next level upon completing specific units and topics.
    \item Clearly display the user's progress as they complete units and topics.
    \item Conduct an initial assessment of the user upon entry to determine their level of knowledge.
    \item Implement a user registration system to track individual progress.
    \item Record and assign points to users for their participation and performance in the application.
    \item Log and display user activity streaks.
    \item Implement different types of assessments, such as quizzes, to measure user progress and performance.
    \item  Generate and display detailed statistics on user progress, including metrics like learning time and assessment results.
\end{enumerate}

\chapter{Other Nonfunctional Requirements}

\section{Performance Requirements}
Geogef prioritizes performance optimization to provide a smooth and responsive learning experience, even during periods of high user traffic.

\section{Security Requirements}
Geogef implements robust security measures to protect user data and ensure secure access to the platform, including user authentication and data encryption protocols.

\section{Software Quality Attributes}
Geogef undergoes rigorous testing and continuous improvement to maintain high standards of quality and reliability, with regular updates based on user feedback and technological advancements.

\section{Business Rules}
Geogef serves as a comprehensive learning platform, offering valuable educational resources and tools to enhance Geography learning outcomes for users worldwide.

\chapter{Other Requirements}

Geogef will require ongoing maintenance and updates to adapt to evolving user needs and technological advancements, ensuring its continued relevance and effectiveness in the field of Geography education.

\end{document}
