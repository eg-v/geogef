\documentclass{article}
\usepackage[utf8]{inputenc}

\title{Práctico 1 de Ingeniería de Software}
\author{Esteban Galucci, Florencia Hernandez, Gonzalo Trepin}
\date{}

\begin{document}

\maketitle

\begin{enumerate}

    \item
    \textbf{En cuanto a la planeación}, tuvimos una etapa de \textit{product discovery}, donde decidimos qué íbamos a construir, planteando las ideas en Miro con un diagrama. Además, definimos los requisitos funcionales y no funcionales de ese sistema de manera detallada.

    \textbf{Para la organización}, particionamos el proyecto en tareas en un \textit{backlog} y las organizamos en un gestor de tareas como Jira de acuerdo a su dificultad e importancia.

    \textbf{La supervisión} era semanalmente los días viernes en el taller, donde Franco analizaba nuestros avances, ayudaba con dudas y nos encaminaba para que el proyecto fuera por buen camino. También recibimos \textit{feedback} del código, donde se sugirieron cambios o mejoras de buenas prácticas. En grupo nos auto-supervisamos manteniendo un control de \textit{commits} con la \textit{pull request} en \textit{GitHub}, donde cada miembro del equipo revisaba lo que quería subir el otro de código.

    \textbf{Sobre el control del desarrollo del proyecto}, revisamos continuamente la documentación del código, controlamos que el código tuviera estructura prolija y legible. Además, cada actualización en el código estaba enfocada en una sola cosa, algo específico.

    \item
    \textbf{Asignación de roles}: Cada miembro del equipo fue asignado a roles específicos basados en sus fortalezas y habilidades. Esto permitió que cada tarea se ejecutara de manera eficiente y que todos estuvieran claros sobre sus responsabilidades.

    \textbf{Comunicación}: Mantuvimos una comunicación constante dentro del equipo, utilizando herramientas como WhatsApp y Discord para coordinar tareas, resolver problemas rápidamente, y asegurar que todos estuvieran alineados con los objetivos del proyecto.

    \textbf{Revisión y \textit{Feedback} Continuo}: Se implementó un sistema de revisiones de código a través de \textit{pull requests} en \textit{GitHub}. Cada miembro del equipo revisaba el código de otros antes de su integración al proyecto principal, lo que ayudó a detectar errores y mejorar la calidad del código desde el inicio.

    \item
    \textbf{Personas}: El equipo de trabajo, Esteban Galucci, Gonzalo Trepin y Florencia Hernandez, y el equipo docente, que se encargaba de supervisar el proyecto.

    \textbf{Procesos}: Seguimos un proceso de desarrollo ágil, llamado \textit{Scrum}, que se enfoca en la entrega iterativa e incremental del producto. En el cual teníamos la planificación del \textit{Sprint} (Seleccionamos las tareas del \textit{backlog} para trabajar en un ciclo corto), el \textit{Daily Standup} (Reuniones diarias breves donde discutimos el progreso, impedimentos y tareas del día), el desarrollo (Durante el \textit{sprint}, trabajamos en las tareas seleccionadas), la revisión del \textit{Sprint} (Al final del \textit{sprint}, presentamos el incremento del producto terminado y recibimos \textit{feedback}) y por último la retrospectiva (Reflexionamos sobre lo que funcionó bien y lo que puede mejorar para los próximos \textit{sprints}).

    \textbf{Productos}: El proyecto fue principalmente guiado por el equipo docente, quienes de alguna manera se encargaban de dejarnos los objetivos de avance para el \textit{sprint} y las herramientas que debíamos utilizar y luego eran supervisados la semana siguiente por ellos mismos. En medio del \textit{sprint}, la organización era completamente nuestra, donde teníamos que dividir tareas que lo hacíamos en función a con qué nos sentimos más cómodos, pero tratando de estar todos en todos lados para no perdernos en el proceso de desarrollo. El producto que logramos fue nuestra aplicación, de alta calidad, llamada \textit{Geogef}, que tenía como objetivo principal la enseñanza de contenidos de geografía del mundo, como banderas, capitales, etc., y logramos construir un producto que pudo satisfacer esas necesidades.

    \item
    Coincidimos en que no realizamos tareas acerca de la gestión del personal, ya que no lo consideramos como parte necesaria del proyecto, pero podrían haber ocurrido algunos problemas relacionados al personal, como por ejemplo que uno del equipo dejara la carrera en medio del proyecto.

    Una de las formas de la gestión de riesgos fue controlar los cambios propuestos al repositorio con ramas nuevas para poder ver las modificaciones antes de agregarlas al proyecto. De esta manera, el equipo de trabajo podía comprobar la existencia de errores en el código y, de esta manera, encontrar errores durante la ejecución. Luego, al probar los cambios y no encontrar algún defecto, los cambios se \textit{mergeaban} en la rama principal.

    \item
    Formamos un equipo de trabajo en torno a un objetivo común claro, como el desarrollo de un software específico, en este caso, \textit{Geogef}. Cada uno de nosotros tenía un rol específico, donde teníamos tareas a cumplir para el avance común del equipo. A diferencia de un grupo, que puede tener objetivos más generales o diversos, un equipo tiene una meta compartida que alinea los esfuerzos y recursos hacia un propósito concreto.

    \item
    Sí, logramos las 5 C. Nos complementamos porque no teníamos problemas a la hora de la asignación de tareas, ya que a cada uno se le facilitaban distintas actividades.

    En el equipo había muy buena comunicación, donde logramos coordinarnos; utilizábamos altamente herramientas de comunicación como WhatsApp y Discord para dividir tareas, compartir avances y proponer ideas nuevas del proyecto.

    La confianza fue un factor elemental en el equipo, debido a que muchas veces estábamos ajustados con las tareas y el tiempo, y confiar en las habilidades y responsabilidad de los otros miembros para que cumplieran sus tareas de forma correcta era esencial.

    El compromiso no faltó en ningún momento, ya que todos participamos de manera activa y continua, lo cual se reflejó en los \textit{commits} del proyecto.

\end{enumerate}

\end{document}

